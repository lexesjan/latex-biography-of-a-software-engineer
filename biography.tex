\documentclass{article}

\usepackage{url}

\title{Biography of a Software Engineer --- Ken Thompson} \author{Lexes Jan
Mantiquilla} \date{\today}

\begin{document}

\maketitle

\section{Introduction}
Kenneth Lane Thompson, also known as Ken Thompson, was born on the 4th of
February 1943 in New Orleans, Loiusiana, U.S. During Thompson's childhood, he
was always facinated with logic and electronics. Thompson got his first
exposure with programming during his junior semester year in lab classes
where they had a G15 computer. Thompson created scaling programs on the G15
computer which was written in assembly.\cite{seibel2009coders}

Thompson attended the University of California, Berkley, graduating with a
Bachelor of Science in 1965 and a Master's degree in Electrial Engineering and
Computer Science in 1966. Following this, Thompson was then hired at Bell
Laboratories. It was here where many of his well known accomplishments were
created, notably Unix, along with Dennis Ritchie, the B programming language and
the UTF-8 character encoding. Thompson retired from Bell Laboratories on the
1st of December, 2000.\cite{linfo} From 2006 onwards, Thompson works at Google
as a distinguished engineer.

\section{Unix Operating System}
One of Thompson's greatest contributions to computer science was the creation
of the Unix operating system along with Dennis Ritchie. Unix is a group of
multitasking, multiuser computer operating systems that was developed by Ken
Thompson, Dennis Ritchie and others at the Bell Laboratories research center.

The first version of Unix was born from out of Thompson's need to test out his
new disk scheduling algorithm's throughput. In order to complete this test, he
required three programs: an editor, an assembler and a kernel. The combination
of these three programs, then, was what created the first version of
Unix.\cite{VCF} The initial versions of the Unix operating system was written
using assembly language, however in 1973, the C language has become
sufficiently mature and Unix was rewritten in C.\cite{ritchie1979evolution}

The popularity of Unix rose between the academic circles and flavours of the
Unix operating system were created most notably the BSD (Berkley Standard
Distribution) version of Unix. Many extensions to Unix have been made in this
flavor of Unix the BSD version of Unix added the TCP/IP protocol support in
1983\cite{unixorg} which gave birth to the world wide web and Bill Joy's vi
editor was also made. At this time AT\&T were also developing the commercial
flavor of Unix called System III and later System V. Many other commercial
companies created their own flavours of Unix such as Sun's Solaris and Apple
Unix which was a BSD based Unix operating system.

\subsection{Impact}
The Unix operating system without a doubt had a big impact on modern day
computing and software engineering. Unix has provided a strong basis for the
Linux kernel. Many Linux and ``Unix-like'' based operating systems power many
devices nowadays such as Android, servers, supercomputers, and mainframes. Most
of the internet today is powered by Linux.

The use of a high level language to write an operating system was unheard of at
time during the creation of Unix. The Unix C rewrite revolutionized operating
systems as it allowed for portability i.e. operating systems were no longer
bound to specific hardware.

The development of Unix lead to the creation of the GNU (GNU's not Unix)
project lead by Richard Stallman. The creation of the GNU project also spawned
the creation of the Free Software Foundation which provides awareness about the
Free Software Movement and is the creator of the popular GPL (GNU General
Public License), a license used in many software projects.

\section{Programming Languages}
\subsection{B}
Thompson has worked on the B programming language during the early versions of
the Unix operating system. Thompson used TMG (TransMoGrifier), created by
Robert M. McClure, to create the compiler for B.  Thompson attempted to
implement a Fortran compiler using TMG, however due to hardware limitations,
many features of the language was stripped and instead B was created as a
result.\cite{VCF}

\subsubsection{Impact}
B is the direct predecessor of C. Dennis Ritchie created C by improving and
extending the functionality of the B programming language, most notably
introducing the types system. C is one of the most influencial and powerful
languages ever created. C is still in wide use today especially for systems
programming such as operating systems and databases programming. C influenced
many languages such as Go, C++, JavaScript, Java and many more.

\subsection{Go}
Go is a compiled and statically typed language created by Google and developed
by Ken Thompson, Robert Griesemer and Rob Pike. Go was created to be an
alternative to C++ and Java and is used to solve many distributed and large
scale problems at Google.

\subsubsection{Impact}
Go is currently one of the fastest growing programming languages. It is the
12th most used programming language according to the 2020 Stack Overflow
Developer Survey.\cite{dev2020survey}

\subsubsection*{Some examples of influential technology created in Go are:}
\begin{itemize}
  \item Docker allows for the ease of deployment through the use of Docker
    containers. Docker has many advantages and the main one is portability. If
    the application deploys successfully on a dev machine, it will deploy on
    the host machine. Other advantages include security and isolation. Since
    the containers are completely isolated, no Docker container can access
    other Docker container processes.
  \item Kubernetes is container orchestration software created by Google. It
    allows for the management, deployment and scaling of containers. Kubernetes
    can be used in conjunction with Docker to allow ease of management of the
    deployment multiple Docker containers to allow for automated redundancy,
    load balancing and provisioning of containers.
\end{itemize}

\section{UTF-8 Character Encoding}
Thompson along with Rob Pike created the first implementation of the UTF-8
encoding scheme for the Plan 9 operating system in 1992.\cite{pikehello} This
encoding scheme is backwards compatible with ASCII.

\subsection{Impact}
As of 2020, the UTF-8 character encoding scheme is the most used encoding
scheme for websites. Over 95.8\% of websites use this
encoding.\cite{web2020tech} This encoding scheme is the reason why we are able
to see websites with multiple languages on the same webpage.

\section{Conclusion}
It is evident from the above arguments that Ken Thompson is a key software
engineer and has many substantial constributions to the world of software
engineering and computing.

\bibliographystyle{plain}
\bibliography{bibliography.bib}

\end{document}

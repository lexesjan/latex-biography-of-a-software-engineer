\documentclass{article}

\usepackage{url}

\title{Biography of a Software Engineer - Ken Thompson}
\author{Lexes Jan Mantiquilla}
\date{\today}

\begin{document}

\maketitle

\section{Introduction}
Kenneth Lane Thompson, also known as Ken Thompson, was born on the 4th of
February 1943 in New Orleans, Loiusiana, U.S. During Thompson's childhood, he
was always facinated with logic and electronics. Thompson got his first
exposure with programming during his junior semester year in lab classes
where they had a G15 computer. Thompson created scaling programs on the G15
computer which was written in assembly. \cite{seibel2009coders}

Thompson attended the University of California, Berkley, graduating with a
Bachelor of Science in 1965 and a Master's degree in Electrial Engineering and
Computer Science in 1966. Following this, Thompson was then hired at Bell
Laboratories. It was here where many of his well known accomplishments were
created, notably Unix, along with Dennis Ritchie, the B programming language and
the UTF-8 character encoding. Thompson retired from Bell Laboratories on the
1st of December, 2000. \cite{linfo} From 2006, Thompson works at Google as a
distinguished engineer.

\section{Unix Operating System}
One of Thompson's greatest contributions to computer science was the creation
of the Unix operating system along with Dennis Ritchie. Unix is a group of
multitasking, multiuser computer operating systems that was developed by Ken
Thompson, Dennis Ritchi and others at the Bell Laboratories research center.

The first version of Unix was born from out of Thompson's need to test out his
new disk scheduling algorithm's throughput. In order to complete this test, he
required three programs: an editor, an assembler and a kernel. The combination
of these three programs, then, was what created the first version of Unix.
\cite{VCF} The initial versions of the Unix operating system was written using
assembly language, however in 1973, the C language has become sufficiently
mature and Unix was rewritten in C. \cite{ritchie1979evolution}

The popularity of Unix rose between the academic circles and flavours of the
Unix operating system were created most notably the BSD (Berkley Standard
Distribution) version of Unix. Many extensions to Unix have been made in this
flavor of Unix the BSD version of Unix added the TCP/IP protocol support in
1983 \cite{unixorg} which gave birth to the world wide web and Bill Joy's vi
editor was also made. At this time AT\&T were also developing the commercial
flavor of Unix called System III and later System V. Many other commercial
companies created their own flavours of Unix such as Sun's solaris and Apple
Unix which was a BSD based Unix operating system.

\subsection{Impact}
The Unix operating system without a doubt had a big impact on modern day
computing and software engineering. Unix has provided a strong basis for the
Linux kernel. Many Linux and ``Unix-like'' based operating systems power many
devices nowadays such as Android, servers, supercomputers, and mainframes. Most
of the internet today is powered by Linux.

The use of a high level language to write an operating system was unheard of at
time during the creation of Unix. The Unix C rewrite revolutionized operating
systems as it allowed for portability i.e. operating systems were no longer
bound to specific hardware.

The development of Unix lead to the creation of the GNU (GNU's not Unix)
project lead by Richard Stallman. The creation of the GNU project also spawned
the creation of the Free Software Foundation which provides awareness about the
Free Software Movement and is the creator of the popular GPL (GNU General
Public License), a license used in many software projects.


\section{Programming Languages}
\subsection{B}
\subsection{Go}
\subsection{Impact}

\section{UTF-8 Character Encoding}
\subsection{Impact}

\section{Conclusion}

\bibliographystyle{plain}
\bibliography{bibliography.bib}

\end{document}
